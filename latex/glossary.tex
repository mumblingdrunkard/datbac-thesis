% this file is required for the template to work, feel free to change
% any definitions, remove as many as you like, add as many as you'd like, but
% don't delete the file.

\newglossaryentry{golang} {
    name=Go,
    description={Statically typed, compiled programming language designed by Robert Griesemer, Rob Pike, and Ken Thompson}
}

\newglossaryentry{c} {
    name=C,
    description={Programming language as old as time itself}
}

\newglossaryentry{java} {
    name=Java,
    description={Object-oriented programming language, compiled to bytecode and run on a virtual machine}
}

\newglossaryentry{cpp} {
    name=C++,
    description={A programming language from the 1980s that was originally based on C}
}

\newglossaryentry{emulator} {
    name=Processor emulator,
    description={A program that emulates the behaviour of a defined processor architecture}
}

\newglossaryentry{branch} {
    name=Branching,
    description={A divergence in the flow of the program}
}

\newglossaryentry{fence} {
    name=Memory fence,
    description={Synchronisation primitive}
}

\newglossaryentry{frame} {
    name=Memory frame,
    description={Unit of physical memory used in paged memory systems}
}

\newglossaryentry{page} {
    name=Memory page,
    description={Unit of virtual memory used in paged memory systems}
}

\newglossaryentry{paging} {
    name=Memory paging,
    description={Technique for virtualising memory}
}

\newglossaryentry{fetchexecute} {
    name=Fetch-execute cycle,
    description={The core functionality of a processor}
}

\newglossaryentry{pipelining} {
    name=Pipelining,
    description={Carefully ordering execution so that multiple operations can be in flight simultaneously}
}

\newglossaryentry{lrsc} {
    name=Load-reserved/store-conditional (load-linked/store-conditional),
    description={Atomic primitives, popular in RISC architectures such as RISC-V or ARM}
}

\newglossaryentry{arm} {
    name=ARM,
    description={Family of RISC instruction set architectures}
}

\newglossaryentry{driver} {
    name=Device driver,
    description={Small program that handles the operation of an external device}
}

\newglossaryentry{preemption} {
    name=Preemption,
    description={Interrupting a process to perform some other task}
}

\newglossaryentry{scheduler} {
    name=Process scheduler,
    description={Manages which processes are running, waiting or blocked in an OS}
}

\newglossaryentry{instruction} {
    name=CPU instruction,
    description={Unit of work in a processor}
}

\newglossaryentry{trap} {
    name=Processor trap,
    description={Mechanism for privilege escalation from user-mode to kernel-mode and enables the handling of exceptions}
}

\newglossaryentry{satp} {
    name=satp,
    description={Supervisor Address Translation and Protection is a control status register for controlling virtual addressing modes and configuration}
}

\newglossaryentry{mtval} {
    name=mtval,
    description={Machine Trap Value is a control status register containing information about the value that caused a trap}
}

\newglossaryentry{mepc} {
    name=mepc,
    description={Machine Exception Program Counter is a control status register containing the value of the program counter as it was before the trap occured}
}

\newglossaryentry{mcause} {
    name=mcause,
    description={Machine Cause is a control status register containing the cause of the trap}
}

\newglossaryentry{mhartid} {
    name=mhartid,
    description={Machine Hart ID is a control status register containing the identity of this RISC-V hart/core}
}

\newglossaryentry{alignment} {
    name=Aligned access,
    description={Access to a data type through an address that is a multiple of the size of the type; e.g. a four-byte int accessed through address 12 would be aligned}
}

\newglossaryentry{word} {
    name=Word,
    description={A sequence of four bytes}
}

\newglossaryentry{byte} {
    name=Byte,
    description={The smallest addressible size of data (eight bits for most modern systems)}
}

\newglossaryentry{short} {
    name=Short,
    description={A sequence of two bytes}
}

\newglossaryentry{livelock} {
    name=Livelock,
    description={State where threads of a process are changing state, but not making any progress; harder to detect than deadlock}
}

\newglossaryentry{deadlock} {
    name=Deadlock,
    description={State where all threads of a process are waiting for a resource held by a different thread, causing the program to freeze}
}
